% Options for packages loaded elsewhere
\PassOptionsToPackage{unicode}{hyperref}
\PassOptionsToPackage{hyphens}{url}
%
\documentclass[
]{article}
\usepackage{amsmath,amssymb}
\usepackage{lmodern}
\usepackage{iftex}
\ifPDFTeX
  \usepackage[T1]{fontenc}
  \usepackage[utf8]{inputenc}
  \usepackage{textcomp} % provide euro and other symbols
\else % if luatex or xetex
  \usepackage{unicode-math}
  \defaultfontfeatures{Scale=MatchLowercase}
  \defaultfontfeatures[\rmfamily]{Ligatures=TeX,Scale=1}
\fi
% Use upquote if available, for straight quotes in verbatim environments
\IfFileExists{upquote.sty}{\usepackage{upquote}}{}
\IfFileExists{microtype.sty}{% use microtype if available
  \usepackage[]{microtype}
  \UseMicrotypeSet[protrusion]{basicmath} % disable protrusion for tt fonts
}{}
\makeatletter
\@ifundefined{KOMAClassName}{% if non-KOMA class
  \IfFileExists{parskip.sty}{%
    \usepackage{parskip}
  }{% else
    \setlength{\parindent}{0pt}
    \setlength{\parskip}{6pt plus 2pt minus 1pt}}
}{% if KOMA class
  \KOMAoptions{parskip=half}}
\makeatother
\usepackage{xcolor}
\IfFileExists{xurl.sty}{\usepackage{xurl}}{} % add URL line breaks if available
\IfFileExists{bookmark.sty}{\usepackage{bookmark}}{\usepackage{hyperref}}
\hypersetup{
  hidelinks,
  pdfcreator={LaTeX via pandoc}}
\urlstyle{same} % disable monospaced font for URLs
\usepackage[margin=1in]{geometry}
\usepackage{graphicx}
\makeatletter
\def\maxwidth{\ifdim\Gin@nat@width>\linewidth\linewidth\else\Gin@nat@width\fi}
\def\maxheight{\ifdim\Gin@nat@height>\textheight\textheight\else\Gin@nat@height\fi}
\makeatother
% Scale images if necessary, so that they will not overflow the page
% margins by default, and it is still possible to overwrite the defaults
% using explicit options in \includegraphics[width, height, ...]{}
\setkeys{Gin}{width=\maxwidth,height=\maxheight,keepaspectratio}
% Set default figure placement to htbp
\makeatletter
\def\fps@figure{htbp}
\makeatother
\setlength{\emergencystretch}{3em} % prevent overfull lines
\providecommand{\tightlist}{%
  \setlength{\itemsep}{0pt}\setlength{\parskip}{0pt}}
\setcounter{secnumdepth}{-\maxdimen} % remove section numbering
\usepackage{fancyhdr}
\pagestyle{fancy}
\lhead{\includegraphics[width = .5\textwidth]{logo.png}}
\ifLuaTeX
  \usepackage{selnolig}  % disable illegal ligatures
\fi

\author{}
\date{\vspace{-2.5em}}

\begin{document}

~

\textbf{Estudi}: Grau en Llengües Aplicadas

\textbf{Nom de l'assignatura:} Mètodes Empírics per a l'Estudi del
Llenguatge 2

\textbf{Codi}: 25667

\textbf{Docent Responsable:} Dr.~Thomas Brochhagen

\textbf{Convocatòria:} Recuperació ~~~ \textbf{Data de l'examen:}
11/07/2022

\textbf{Curs:} 3er ~~~\textbf{Trimestre:} 3er

\textbf{Nom i cognoms}:

\begin{center}\rule{0.5\linewidth}{0.5pt}\end{center}

\hypertarget{fundamentos-1-punto-por-pregunta-20-total}{%
\section{Fundamentos (1 punto por pregunta / 20
total)}\label{fundamentos-1-punto-por-pregunta-20-total}}

\begin{enumerate}
\def\labelenumi{\arabic{enumi}.}
\tightlist
\item
  Da un ejemplo de una muestra en la cual el promedio y la mediana son
  iguales
\item
  Da un ejemplo de una muestra en la cual el promedio es mayor que la
  mediana
\item
  Qué tipo de variable es la edad de un participante en un experimento?
  Justifica tu respuesta
\item
  Qué tipo de variable es el idioma materno de un participante en un
  experimento? Justifica tu respuesta
\item
  Qué tipo de variable es la longitud de una palabra? Justifica tu
  respuesta
\item
  Una muestra no-representativa, puede ser completa? Justifica tu
  respuesta
\item
  Da un ejemplo de un caso en el cual utilizarías el promedio para
  resumir una muestra, en comparación con la mediana; y vice-versa.
  Justifica tu respuesta
\item
  Da un ejemplo de un caso en el cual utilizarías la desviación estándar
  para resumir una muestra, en vez de la varianza. Justifica tu
  respuesta
\item
  ¿Cuáles son los parámetros de la distribución Gaussiana? Explica, de
  manera intuitiva, qué hace cada parámetro
\item
  ¿Cuáles son los parámetros de la distribución Binomial? Explica, de
  manera intuitiva, qué hace cada parámetro
\item
  ¿Cuáles son los parámetros de la distribución de Poisson? Explica, de
  manera intuitiva, qué hace cada parámetro
\item
  ¿De qué manera se relacionan el tamaño de una muestra y el efecto que
  se estima a base de ella?
\item
  ¿En qué sentido es lineal una regresión lineal?
\item
  ¿Qué indica \(R^2\)?
\item
  Menciona tres diferencias entre \(R^2\) y el Akaike Information
  Criterion (AIC)
\item
  Menciona tres maneras en las cuales se puede pre-procesar un corpus de
  texto antes de analizarlo
\item
  La predicción de una regresión de Poisson, ¿puede ser 0? Justifica tu
  respuesta
\item
  La predicción de una regresión de Bernoulli, ¿puede ser 5? Justifica
  tu respuesta
\item
  La predicción de una regressión Normal, ¿puede ser 0? Justifica tu
  respuesta
\item
  Menciona una diferencia entre métodos estadísticos descriptivos e
  inferenciales
\end{enumerate}

\hypertarget{interpretaciuxf3n-2-puntos-por-pregunta-20-total}{%
\section{Interpretación (2 puntos por pregunta / 20
total)}\label{interpretaciuxf3n-2-puntos-por-pregunta-20-total}}

\begin{enumerate}
\def\labelenumi{\arabic{enumi}.}
\tightlist
\item
  Un ejemplo de una muestra (no) representativa
\end{enumerate}

\hypertarget{aplicaciuxf3n-5-puntos-por-pregunta-30-total}{%
\section{Aplicación (5 puntos por pregunta / 30
total)}\label{aplicaciuxf3n-5-puntos-por-pregunta-30-total}}

\hypertarget{cruxedtica-5-puntos-por-pregunta-30-total}{%
\section{Crítica (5 puntos por pregunta / 30
total)}\label{cruxedtica-5-puntos-por-pregunta-30-total}}

\end{document}
